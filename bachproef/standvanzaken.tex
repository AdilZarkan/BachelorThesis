\chapter{\IfLanguageName{dutch}{Stand van zaken}{State of the art}}
\label{ch:stand-van-zaken}

% Tip: Begin elk hoofdstuk met een paragraaf inleiding die beschrijft hoe
% dit hoofdstuk past binnen het geheel van de bachelorproef. Geef in het
% bijzonder aan wat de link is met het vorige en volgende hoofdstuk.

% Pas na deze inleidende paragraaf komt de eerste sectiehoofding.

Dit hoofdstuk bevat je literatuurstudie. De inhoud gaat verder op de inleiding, maar zal het onderwerp van de bachelorproef *diepgaand* uitspitten. De bedoeling is dat de lezer na lezing van dit hoofdstuk helemaal op de hoogte is van de huidige stand van zaken (state-of-the-art) in het onderzoeksdomein. Iemand die niet vertrouwd is met het onderwerp, weet nu voldoende om de rest van het verhaal te kunnen volgen, zonder dat die er nog andere informatie moet over opzoeken \autocite{Pollefliet2011}.

Je verwijst bij elke bewering die je doet, vakterm die je introduceert, enz. naar je bronnen. In \LaTeX{} kan dat met het commando \texttt{$\backslash${textcite\{\}}} of \texttt{$\backslash${autocite\{\}}}. Als argument van het commando geef je de ``sleutel'' van een ``record'' in een bibliografische databank in het Bib\LaTeX{}-formaat (een tekstbestand). Als je expliciet naar de auteur verwijst in de zin, gebruik je \texttt{$\backslash${}textcite\{\}}.
Soms wil je de auteur niet expliciet vernoemen, dan gebruik je \texttt{$\backslash${}autocite\{\}}. In de volgende paragraaf een voorbeeld van elk.

%\textcite{Knuth1998} schreef een van de standaardwerken over sorteer- en zoekalgoritmen. Experten zijn het erover eens dat cloud computing een interessante opportuniteit vormen, zowel voor gebruikers als voor dienstverleners op vlak van informatietechnologie~\autocite{Creeger2009}.

\section{Microsoft Dynamics 365 for Finance and Operations}
\subsection{Voorgeschiedenis van MS Dynamics 365 for Finance and Operations}
Dynamics 365 in de huidige naam en vorm is pas recent tot stand gekomen. De suites geschiedenis begint al in de jaren 80. Personal computing begon meer en meer toegankelijker te worden, en dit zagen ook veel bedrijven in. Zelfs toen al begon men  computers te gebruiken om hun financiën en operaties te beheren. In \textcite{Olsen et al.2009} stelt men dat meer dan 25 jaar aan ervaring in business applicaties en ontwikkel productiviteit bevat. Daarom is het nodig om eerst een goed beeld te scheppen over de rijke geschiedenis van het ERP-systeem zoals beschreven per \textcite{Wright2018}

\subsubsection{Solomon Software }
In 1991 bracht Solomon Software hun Solomon IV uit. Dit was oorspronkelijk begonnen als Solomon I die vooral boekhoudsoftware was, om uiteindelijk te groeien naar de IV-versie die speciaal ontworpen was om met het toen nog relatief nieuwe Microsoft Windows operating system te werken. Dit product maakte veel furore in het technologielandschap dankzij de flexibiliteit, en kwalitatieve kenmerken die de software toen al bezat. 


\subsubsection{Great Plains }
In 1991 bracht Solomon Software hun Solomon IV uit. Dit was oorspronkelijk begonnen als Solomon I die vooral boekhoudsoftware was, om uiteindelijk te groeien naar de IV-versie die speciaal ontworpen was om met het toen nog relatief nieuwe Microsoft Windows operating system te werken. Dit product maakte veel furore in het technologielandschap dankzij de flexibiliteit, en kwalitatieve kenmerken die de software toen al bezat. 

In 2001 kocht Microsoft Great Plains over, inclusief hun rechten over Solomon IV, en beschikte het zo ook over Dynamics (inmiddels al Release 8). In het zelfde jaar kocht de technologiegigant ook nog een software uit Virginia, genaamd ICommunicate, de makers van een web-gebaseerd CRM-programma gekend als ICommunicate.NET. Aanvankelijk dacht men bij Microsoft dat de fundering van deze software en bedrijven genoeg zou zijn om hun eigen Business Solutions afdeling, en dus een ERP van Microsoft, op te starten. Helaas bleek de Dynamics software gemaakt door Great Plains hier te inadequaat voor. 

\subsubsection{Internationale partners }
Dus ging Microsoft op zoek naar nieuwe partner, en deze vonden ze in Copenhagen, Denemarken. Het bedrijf PC\&C bracht in 1984 hun boekhoudsoftware PC Plus uit. 3 jaar later kwam een multi-user variant hiervan uit die was ontwikkeld in samenwerking met IBM Denemarken, deze versie zou bekend worden onder de naam Navigator. In 1995 kwam de eerste Windows-gebaseerde versie uit, onder de naam Navision Financials. PC\&C zou uiteindelijk fusioneren met een ander bedrijf, namelijk Damgaard, die hun eigen ERP waren aan het ontwikkelen. Zo bracht Damgaard in 1998 hun eerste versie van Axapta uit, een oplossing voor financieel-, voorraad-, en productiebeheer.
 
Deze fusie tussen PC\&C en Axapta duurde 2 jaar alvorens Microsoft beide bedrijven in 2002 zou overkopen, om ze samen onder hun vleugels te krijgen samen met de Amerikaanse bedrijven Great Plains,  Solomon Software en ICommunicate. 

\subsubsection{Business Solutions afdeling}
Nu Microsoft al deze bedrijven had overgekocht, beschikte het over de nodige bouwblokken en know-how om hun “Microsoft Business Solutions Division” op te starten. 

In de eerste jaren bleef Microsoft geüpdatet versies uitbrengen van hun overgenomen software programma’s. updates bestonden vooral uit het toevoegen van role-based interfaces, SQL-gebaseerde rapporteringen, en integraties met onder andere Office en SharePoint.  Zo kwam in 2002 MBS Axapta uit, vervolgens volgden ook MBS Icommunicate.NET en MBS Customer Relationship Management uit in 2003. Tenslotte kwam in 2004 ook MBS Great Plains voor het eerst uit. 

\subsubsection{Project Green }
Echter was dit niet het hoofddoel van Microsoft. Het plan daarentegen was om al deze programma’s te consolideren onder 1 “super-oplossing”. Deze doelstelling werd belichaamd onder de naam Project Green. Hier begon men aan te werken in 2003, met als belofte om in 2004 een eerste beta-versie uit te brengen. 

Al bleek dit veel gemakkelijker gezegd dan gedaan. Het samenbrengen van al deze systemen bleek niet zo eenvoudig als gedacht voor de technologiegigant, zeker als men in rekening houdt dat het op dat moment zijn eerste stappen was aan het zetten in de ERP industrie. Microsoft begon de oplossingen gradueel aan te passen, voornamelijk beginnend met de user interfaces om deze meer overeen te stemmen met de reeds bestaande Microsoft producten zoals Office en Outlook. 


\subsection{Microsoft Dynamics wordt geboren}
In 2006 gaat Microsoft een stap verder. De ERP-afdeling (voordien gekend als MBS) krijgt een nieuwe naam en wordt Microsoft Dynamics. Ook MBS Navision werd MS Dynamics NAV, Axapta werd MS Dynamics AX, MBS Great Plains werd MS Dynamics GP, MBS Solomon werd MS Dynamics SL en ook MS CRM kreeg een naamsverandering naar MS Dynamics CRM. 
Tegen 2007 was Project Green uitgestorven. De systemen bleken te complex om samen te voegen onder 1 mega-oplossing. De officiële aankondiging van Microsoft was dat men de verschillende Dynamics platformen individueel ging blijven ontwikkelen om zo goed mogelijk te kunnen antwoorden op specifieke klantennoden. 


\subsection{Cloud-first, mobile-first}
Nadat men Project Green had laten uitsterven, begon men te focussen op een nieuwe factor in het toenmalig technologielandschap: cloud-computing.  
Zo bracht Microsoft in 2007 hun eerste online business solutions client uit. Dynamics CRM was een web-hosted versie van de reeds bestaande software, en men kon toegang krijgen tot het platform dankzij Microsoft’s dedicated CRM Online Service, of via een Microsoft partner. Deze stap was cruciaal om flexibelere en meer toegankelijke diensten aan te kunnen bieden voor de Dynamics familie, bevrijd van de beperkingen van on-premise local hosting. 
Daarom hadden tegen 2013 ook Dynamics GP en Dynamics NAV web interfaces, en in 2016 kwamen al eerste apps uit voor sommige van de Dynamics platformen. Dit maakte de software-suite alleen maar toegankelijker en gebruiksvriendelijker. Verder werd in 2013 ook de Office-suite gere-brand naar Office 365.  


\subsection{Dynamics 365}
5 jaar na Office 365, kreeg ook de Dynamics-afdeling een nieuwe naam. Het zou voortaan Dynamics 365 heten, zoals we het vandaag kennen. Echter bracht deze verandering niet enkel een nieuwe naam met zich mee, maar ook unieke aspecten. Zo werd het volledig omgebouwd naar een nieuwe suite van CRM en ERP apps gecombineerd, verpakt in nieuwe functionaliteiten en een volledig nieuw licentie-model. 
Deze aankondiging werd door de CEO van Microsoft, Satya Nadella, zelf aangekondigd op linkedIn \textcite{Nadella2016}. Deze wijziging was naar eigen zeggen al meer dan 15 jaar een droom, al sinds de overname van Great Plains. Dit allemaal om de gebruikers meer inzichten te geven, meer mogelijkheden te geven om grootste zaken mogelijk te maken. Ook kondigt de CEO aan meer klant-gebaseerde diensten aan te gaan bieden, waarin elke gebruiker zelf kan bepalen welke delen van de omvangrijke Dynamics-suite nodig te hebben, zonder compromissen te hoeven doen in kwaliteit. De Dynamics 365 suite kan niet alleen perfect meegroeien met de noden van de klanten, dit was de hoofddoelstelling toen men de veranderingen begon. 
Tenslotte is de CEO er ook van overtuigd dat deze unieke aanpak van bouwen van technologie een zeer wijdverspreide digitale transformatie tot stand zal brengen die elke bedrijf, in elke industrie en in elk land voelen. Dit zal op zijn beurt dan leiden tot een verhoogde aantal opportuniteiten voor alle Microsoft partners in heel de wereld. 


\section{Chatbots}
\subsection{Wat zijn chatbots?}
\subsubsection{definitie}
De klassieke definitie van Chatbots wordt volgens \textcite{Khan2017} verwoordt als “een computer programma dat natuurlijke taal-input van een gebruiker kan verwerken, om hier vervolgens een op maat gemaakt antwoord aan terug te koppelen”. 

De term “Chatbot” is oorspronkelijk afkomstig van Michael Mauldin, de maker van Verbot. Zo defineerde hij zijn eerste prototype, “Julia®”, als een prototype “chatterbot”. Per \textcite{Vlahos2019} is dit een van de eerste instanties van een chatbot, die dateert uit 1994. Verbot, dat overigens een afkorting is voor “Verbally Enhanced Software Robot”, begon oorspronkelijk als een prototype met als eerste instantie “Julia®”. Dit was een een chatbot die destijds gecreëerd was door mr. Mauldin om deel te nemen aan de internationaal bekende Turing Test om zo mee te dingen voor de prestigieuze Loebner-prijs. Volgens \textcite{Moor2012} kan de Turing-Test gedefinieerd worden als een computer die interageert met een mens, zonder dat de mens kan uitmaken of zijn gesprekspartner een mens is of machine. Op \textcite{O'Neill2016} definieert men de Loebner-prijs als de oudste Turing Test wedstrijd, gestart in 1991. Deze wedstrijd kent elk jaar een prijs uit voor de sterkst Artificieel Intelligent-gerelateerde ontwikkeling van dat jaar. Volgens het boek is geen enkele chatbot hier in geslaagd voor 2014. 
Ook Michael Mauldin was een pioneer op vlak van Chatbots. Zo heeft de computer wetenschapper 2 boeken, en meer dan 5 papers geschreven over Natural Language Processing, en rationale agenten die de basis vormen voor Artificiële Intelligentie. Nadat hij in in 1994 zijn prototype “Julia®” had uitgebracht, werkte de computer wetenschapper er samen met Peter Plantec, een klinische psycholoog, verder aan het project. Zo richtten ze in 1997 samen het bedrijf Virtual Personalities, Inc. Op. De uitkomst van deze samenwerking was “Sylvie®”, die in de herfst van 2000 werd uitgebracht. De insteek hiervan was om een echt virtueel persoon na te bootsen, en dit probeerden ze met behulp van avatars voor de grafische voorstelling, naast spraaktechnologie en Natural Language Processing. 

\section{Voorgeschiedenis}
Een van de eerste instanties van een chatbot, vindt zijn oorsprong terug in 1964. Joseph Weizenbaum \autocite{Weizenbaum1966} begon toen aan zijn project “Eliza” waar hij uiteindelijk meer dan 2 jaar aan zou besteden. “Eliza” onderzocht de kernwoorden van gebruikers hun input, en zorgde zo voor het terugkoppelen van juiste output die de gebruiker op dat moment verwacht. 
“Parry”, een van de volgende chatbots, werd niet veel later uitgebracht door psychiater Kenneth Colby. Dit deed hij in een poging om personen met paranoïde Schizofrenie te stimuleren. 
Tenslotte staat “A.L.I.C.E.”, of “Alicebot”, die in 2000 werd ontwikkeld door Richard Wallace en zich baseerde op “Eliza” nog steeds bekend als een van de meest robuuste chatbots. Ookal slaagde het er niet in om te slagen voor de Turing-Test, won het toch de voorvernoemde Loebner-prijs maar liefst drie keer. “A.L.I.C.E.” is overigens een afkorting voor Artificial Linguistic Internet Computer Entity. 

\section{Natural Language Processing (NLP)}
 \textcite{Seif2018} Definieert Natural Language Processing als een subdomein van Artificiele Intelligentie. Dit domein focust zich op het laten begrijpen, en verwerken, van natuurlijke taal door een computer. Computers hebben immers geen notie van intuïtie, en kunnen dus niet zelfstandig ontcijferen wat een gebruiker bedoelt wanneer hij iets zegt. Zo kan een simpele zin als “ik heb de tegenpartij helemaal verwoest tijdens de voetbalmatch van gisteren”, al snel verkeerd geïnterpreteerd worden door een chatbot. Wanneer het woord verwoest wordt ingelezen kan immers al snel een foute actie gekozen worden, terwijl mensen wel in staat zijn om de zin te ontleden.  

\section{Chatbots vandaag}
Wanneer we naar de geschiedenis kijken, bracht elke speler in de chatbot-markt steeds zijn eigen chatbot uit, voor zijn eigen doelpubliek. In de recente jaren is deze trend echter veranderd. Telegram was een van de eerste platformen die zijn bot platform in 2015 opende voor het grote publiek. Nadat ook Slack in december 2015 mee ging in deze trend, werden veel Techgiganten getriggerd om hierin mee te gaan. Nadat Facebook in april 2016, op hun jaarlijkse F8-conferentie, hun Messenger platform uit de doeken deed kwam de chatbot-trend in een stroomversnelling. De opportuniteit om 1 biljoen gebruikers te bereiken  via Messenger speelde een gigantische rol in de populariteit hiervan. Snel volgden ook Skype, Kik, WeChat en andere Tech giganten. 


\lipsum[7-20]
